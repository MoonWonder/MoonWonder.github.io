\documentclass[UTF8]{ctexart}
\usepackage{amsmath}
\usepackage{amsfonts}
\usepackage{amssymb}
\usepackage{graphicx}
\usepackage{geometry}
\geometry{a4paper, margin=1in}
\usepackage{enumitem} % 用于自定义列表

\title{第35章 干涉 - 课堂笔记 (含例题)}
\author{南方科技大学 普通物理II}
\date{\today}

\begin{document}
\maketitle

\section{光作为波}
\begin{itemize}
    \item 光表现出波的特性。
    \item \textbf{惠更斯原理}:波前的每一点都可以看作是次级球面子波的波源。新的波前是所有这些次级子波的包络面(切面)。
        \begin{itemize}
            \item 这是一种可视化光传播的便捷方法。
            \item 麦克斯韦方程组是电磁波传播的基本关系。
        \end{itemize}
    \item 惠更斯原理可以用来解释折射定律。
    \item 光在折射率为 $n$ 的介质中的波长为 $\lambda_n = \frac{\lambda}{n}$,其中 $\lambda$ 是在真空中的波长。
\end{itemize}

\section{干涉 (Interference)}
\begin{itemize}
    \item 当两个或多个波在空间中重叠时,会发生干涉。
    \item 两束光波发生干涉的条件:
        \begin{enumerate}
            \item \textbf{相干性}:单色光(相同的频率/波长)且具有恒定的相位关系。
            \item 振幅相等。
            \item 偏振方向相同(波的振动在同一平面内)。
        \end{enumerate}
    \item \textbf{干涉与衍射}:
        \begin{itemize}
            \item 两者之间没有本质的区别。都是叠加原理和惠更斯原理的结果。
            \item \textbf{干涉}:通常指涉及来自少量波源(通常是两个)的波的效应。
            \item \textbf{衍射}:通常涉及惠更斯子波在孔径区域的连续分布,或大量波源或孔径。
        \end{itemize}
\end{itemize}

\section{杨氏双缝干涉实验}
\begin{itemize}
    \item 演示了光波通过两条狭缝时的干涉现象。
    \item 相干的单色光照射两条窄缝($S_1$ 和 $S_2$)。
    \item 在屏幕上观察到明暗相间的条纹(干涉图样)。
        \begin{itemize}
            \item \textbf{亮条纹(相长干涉)}:波前同相到达。
            \item \textbf{暗条纹(相消干涉)}:波前反相到达。
        \end{itemize}
    \item \textbf{程差和相位差}:
        \begin{itemize}
            \item 假设到屏幕的距离 $R$ 远大于狭缝间距 $d$($R \gg d$),则光线可以视为平行。
            \item 程差:$\Delta r = r_2 - r_1 = d \sin\theta$。
            \item 相位差:$\Delta\phi = \frac{2\pi}{\lambda} (d \sin\theta)$。
        \end{itemize}
    \item \textbf{条纹条件}(近似几何):
        \begin{itemize}
            \item \textbf{相长干涉(亮条纹)}:
                \begin{itemize}
                    \item $d \sin\theta = m\lambda$,其中 $m = 0, \pm1, \pm2, \dots$
                    \item 相位差:$\Delta\phi = 2m\pi$。
                \end{itemize}
            \item \textbf{相消干涉(暗条纹)}:
                \begin{itemize}
                    \item $d \sin\theta = (m + \frac{1}{2})\lambda$,其中 $m = 0, \pm1, \pm2, \dots$
                    \item 相位差:$\Delta\phi = (2m+1)\pi$。
                \end{itemize}
        \end{itemize}
    \item \textbf{屏幕上条纹的位置}:
        \begin{itemize}
            \item 对于小角度 $\theta$,$\sin\theta \approx \tan\theta \approx \frac{y_m}{R}$。
            \item 亮条纹的位置:$y_m = R \frac{m\lambda}{d}$。
            \item 由此可以测定光的波长:$\lambda = \frac{y_m d}{mR}$。
        \end{itemize}
    \item \textbf{例题1:计算波长}
        \begin{itemize}
            \item 在杨氏双缝干涉实验中,双缝间距 $d = 0.200 \text{ mm}$,屏幕到双缝的距离 $R = 1.00 \text{ m}$。测得第三级亮条纹 ($m=3$) 到中心亮纹的距离为 $y_3 = 9.49 \text{ mm}$。求入射光的波长。
            \item \textbf{解}:根据亮条纹位置公式 $y_m = R \frac{m\lambda}{d}$,可得 $\lambda = \frac{y_m d}{mR}$。
            \item 代入数据:$\lambda = \frac{(9.49 \times 10^{-3} \text{ m})(0.200 \times 10^{-3} \text{ m})}{(3)(1.00 \text{ m})} = 632.67 \times 10^{-9} \text{ m} \approx 633 \text{ nm}$。
        \end{itemize}
\end{itemize}

\section{由介质差异引起的相位差}
\begin{itemize}
    \item 当光在不同介质中传播时,在距离 $L$ 内的波长数会发生变化。
    \item 在介质1中的波长数:$N_1 = \frac{L}{\lambda_{n1}} = \frac{Ln_1}{\lambda_a}$。
    \item 在介质2中的波长数:$N_2 = \frac{L}{\lambda_{n2}} = \frac{Ln_2}{\lambda_a}$。
    \item 相位差:$\Delta\phi = 2\pi (N_2 - N_1) = \frac{2\pi L}{\lambda_a} (n_2 - n_1)$。
    \item 这种相位差可以改变干涉条纹的位置。
    \item \textbf{例题:介质移动条纹}
        \begin{itemize}
            \item 考虑将一块厚度为 $L$、折射率为 $n_2$ 的透明介质插入杨氏双缝实验的一条光路中(原光路在空气中,折射率 $n_1 \approx 1$)。如果希望将原先的 $m=0$ 级亮条纹移动到原先 $m=1$ 级亮条纹的位置,介质的厚度 $L$ 应为多少?假设空气中波长为 $\lambda_a$。
            \item \textbf{解}:条纹移动一个条纹间距,相当于引入的相位差为 $2\pi$ (或光程差为一个波长)。
            \item 由介质引入的光程差为 $L(n_2 - n_1)$。
            \item 因此 $L(n_2 - n_1) = 1 \cdot \lambda_a$。
            \item $L = \frac{\lambda_a}{n_2 - n_1}$。如果 $n_1=1$ (空气),则 $L = \frac{\lambda_a}{n_2 - 1}$。
            \item 例如,如果 $\lambda_a = 600 \text{ nm}$,$n_2=1.5$,$n_1=1$,则 $L = \frac{600 \text{ nm}}{1.5-1} = 1200 \text{ nm} = 1.2 \mu\text{m}$。
        \end{itemize}
\end{itemize}

\section{干涉图样的强度}
\begin{itemize}
    \item 假设两个相干波具有相同的振幅 $E_0$ 和相同的偏振方向。
    \item 强度 $I \propto E_0^2$。
    \item 使用相量图来计算电场的叠加:
        \begin{itemize}
            \item $E_1(t) = E_0 \cos(\omega t)$
            \item $E_2(t) = E_0 \cos(\omega t + \phi)$,其中 $\phi = \frac{2\pi d \sin\theta}{\lambda}$。
        \end{itemize}
    \item 合成振幅:$E = 2E_0 \cos(\frac{1}{2}\phi)$。
    \item 合成强度:$I = 4I_0 \cos^2(\frac{1}{2}\phi)$,其中 $I_0$ 是单个狭缝在该处的强度(对于窄缝,常假定处处相等)。
        \begin{itemize}
            \item 最大强度 $I_{max} = 4I_0$。
            \item 能量被重新分配,总量不变。
        \end{itemize}
    \item 强度分布:
        \begin{itemize}
            \item 最大值($\cos^2(\frac{1}{2}\phi) = 1$):$\phi = 2m\pi \implies \text{程差 } \Delta L = m\lambda$。
            \item 最小值($\cos^2(\frac{1}{2}\phi) = 0$):$\phi = (2m+1)\pi \implies \text{程差 } \Delta L = (m+\frac{1}{2})\lambda$。
        \end{itemize}
\end{itemize}

\section{薄膜干涉}
\begin{itemize}
    \item 由于光在薄膜上下表面的反射而产生的干涉现象。
    \item 关键因素:
        \begin{enumerate}
            \item 光在薄膜材料中波长的改变。
            \item 反射时的相位变化。
        \end{enumerate}
    \item \textbf{反射时的相位变化}:
        \begin{itemize}
            \item 如果光从折射率较小的介质入射到折射率较大的介质界面反射($n_a < n_b$),则会发生 $\pi$ 弧度(半个周期)的相位跃变。
            \item 如果光从折射率较大的介质入射到折射率较小的介质界面反射($n_a > n_b$),则没有相位跃变。
        \end{itemize}
    \item \textbf{肥皂膜的反射}(例如,空气-薄膜-空气,$n_1=n_3 < n_2$):
        \begin{itemize}
            \item 光线1(从上表面反射,$n_1 \to n_2$):有 $\pi$ 相位跃变。
            \item 光线2(从下表面反射,$n_2 \to n_3$):无相位跃变(如果 $n_2 > n_3$)。
            \item 薄膜内的程差:$2L$(对于垂直入射 $\theta \approx 0^\circ$)。
            \item 有效程差包括相位跃变的影响。
            \item $r_2$ 和 $r_1$ 之间的相位差:$(\frac{2L}{\lambda_{n2}} - \frac{1}{2}) \cdot 2\pi$。
            \item \textbf{相长干涉}:$2L = (m + \frac{1}{2})\frac{\lambda}{n_2}$,其中 $m=0, 1, 2, \dots$
            \item \textbf{相消干涉}:$2L = m\frac{\lambda}{n_2}$,其中 $m=0, 1, 2, \dots$
            \item 如果 $L \ll \lambda$,由于反射引起的 $\frac{1}{2}\lambda$ 有效程差,总是发生相消干涉。
        \end{itemize}
    \item \textbf{薄膜与厚膜}:
        \begin{itemize}
            \item \textbf{薄膜}:反射波来自同一波列,是相干的。
            \item \textbf{厚膜}:反射波来自不同波列,是不相干的。
        \end{itemize}
    \item \textbf{薄膜干涉的不同情况}(垂直入射):
        \begin{itemize}
            \item \textbf{情况1 ($n_1 < n_2 > n_3$)}:一次 $\pi$ 相位跃变(在界面1)。
                \begin{itemize}
                    \item 相长干涉:$2t = (m + \frac{1}{2})\frac{\lambda}{n}$ (其中 $n$ 是薄膜折射率)。
                \end{itemize}
            \item \textbf{情况2 ($n_1 > n_2 < n_3$)}(例如空气劈尖):一次 $\pi$ 相位跃变(在界面2,即 $n_2 \to n_3$ 时 $n_2 < n_3$)。
                 \begin{itemize}
                    \item 对于空气劈尖($n_1$ 玻璃,$n_2$ 空气,$n_3$ 玻璃),光线在空气膜下表面反射时(空气 $\to$ 玻璃,$n_{空气} < n_{玻璃}$)有 $\pi$ 相位跃变。上表面反射(玻璃 $\to$ 空气)无相位跃变。
                    \item 相长干涉:$2t = (m + \frac{1}{2})\lambda_{空气}$。
                    \item 牛顿环是这种情况的一个例子(空气膜)。
                        \begin{itemize}
                            \item 亮环(中心为暗点后):$2d = (m + \frac{1}{2})\lambda_{空气}$ (其中 $m=0, 1, 2, \dots$ 对应第一亮环,第二亮环等)。
                        \end{itemize}
                \end{itemize}
            \item \textbf{情况3 ($n_1 < n_2 < n_3$)}:两次 $\pi$ 相位跃变(在两个界面)。
                \begin{itemize}
                    \item 相长干涉:$2t = m\frac{\lambda}{n_2}$ (其中 $n_2$ 是薄膜折射率)。
                    \item “增透膜”(四分之一波长光学厚度 - QWOT):特定波长发生相消干涉。若要中心波长 $\lambda_0$ 在膜中相消,则 $2n_2 t = (m+\frac{1}{2})\lambda_0$,通常取 $m=0$,则 $t = \frac{\lambda_0}{4n_2}$。
                \end{itemize}
            \item \textbf{情况4 ($n_1 > n_2 > n_3$)}:没有 $\pi$ 相位跃变。
                \begin{itemize}
                     \item 相长干涉:$2t = m\frac{\lambda}{n_2}$。
                \end{itemize}
        \end{itemize}
    \item \textbf{镀膜}:
        \begin{itemize}
            \item \textbf{增透膜}:利用相消干涉来减少反射,增强透射。厚度通常是特定波长在膜中波长的 $1/4$ 的奇数倍。
            \item \textbf{增反膜}:利用相长干涉来增强反射。厚度通常是特定波长在膜中波长的 $1/4$ 的奇数倍(取决于上下介质的折射率关系,确保同相叠加)。
        \end{itemize}
\end{itemize}

\section{迈克尔逊干涉仪}
\begin{itemize}
    \item 一种利用干涉进行精确测量的仪器。
    \item 将一束光分成两路(使用分束器),分别从反射镜($M_1$ 固定,$M_2$ 可移动)反射,然后重新组合以观察干涉条纹。
    \item 使用补偿板是为了使两束光在玻璃中的光程相等。
    \item 移动反射镜 $M_2$ 会改变程差,导致条纹移动。
    \item 根据反射镜的调整,可以产生“等倾条纹”(当$M_1$和$M_2$的虚像$M_1^*$严格平行时)或“等厚条纹”(当$M_1$和$M_2$的虚像$M_1^*$之间有小角度时)。
\end{itemize}

\section{应用}
\begin{itemize}
    \item \textbf{等倾干涉}:当薄膜厚度均匀,入射角变化时发生,形成圆形条纹。
    \item \textbf{LIGO(激光干涉引力波天文台)}:使用大型迈克尔逊干涉仪探测引力波。
        \begin{itemize}
            \item 2016年2月11日首次探测到引力波,来自于13亿光年之外两个黑洞的合并。
        \end{itemize}
\end{itemize}

\section{关键公式总结}
\begin{itemize}
    \item \textbf{杨氏双缝干涉(亮条纹)}:$d \sin\theta = m\lambda$; $y_m = R \frac{m\lambda}{d}$。
    \item \textbf{相位差(介质)}:$\Delta\phi = \frac{2\pi L}{\lambda_a} (n_2 - n_1)$。
    \item \textbf{强度(两个相干光源)}:$I = 4I_0 \cos^2(\frac{1}{2}\phi)$,其中 $\phi = \frac{2\pi d}{\lambda}\sin\theta$。
    \item \textbf{薄膜干涉(具体条件取决于界面处的相位跃变)}:
        \begin{itemize}
            \item 情况1 ($n_1 < n_2 > n_3$,一次跃变):相长干涉 $2t = (m + \frac{1}{2})\frac{\lambda}{n_2}$。
            \item 情况2 (例如,空气劈尖,$n_1 > n_2 < n_3$,一次跃变):相长干涉 $2t = (m + \frac{1}{2})\lambda_{空气}$。
            \item 情况3 ($n_1 < n_2 < n_3$,两次跃变):相长干涉 $2t = m\frac{\lambda}{n_{film}}$。
        \end{itemize}
\end{itemize}

\newpage % 另起一页放置思考题和答案
\section*{思考题汇总}
\begin{enumerate}[label=\textbf{Q\arabic*.}, leftmargin=*]
    \item 一束波长为 660 nm 的单色光射向一个双缝。考虑图中标出的五个条纹(A, B, C, D, E),包括标记为“B”的中央最大值 ($m=0$)。当程差为 1320 nm 时,会产生哪个条纹?(假定条纹从左到右或从下到上依次为 A, B, C, D, E,且B为中央亮纹)
        \begin{itemize}
            \item[] (a) A (b) B (c) C (d) D (e) E
        \end{itemize}

    \item 在杨氏实验中,通过两条狭缝 $S_1$ 和 $S_2$ 的相干光在远处的屏幕上产生明暗相间的图样。从 $S_1$ 到 $m=+3$ 亮条纹区域的距离与从 $S_2$ 到 $m=+3$ 亮条纹区域的距离之差是多少?
        \begin{itemize}
            \item[] A. 三个波长 B. 三个半波长 C. 四分之三个波长 D. 信息不足,无法判断
        \end{itemize}

    \item 在杨氏实验中,通过两条狭缝 $S_1$ 和 $S_2$ 的相干光在远处的屏幕上产生明暗相间的图样。如果光的波长增加,图样会如何变化?
        \begin{itemize}
            \item[] A. 亮区靠得更近。
            \item[] B. 亮区离得更远。
            \item[] C. 亮区之间的间距保持不变,但颜色改变。
            \item[] D. 以上任何一种情况,取决于具体条件。
            \item[] E. 以上都不是。
        \end{itemize}

    \item 一个空气劈尖分隔两块玻璃板,如图所示。波长为 $\lambda$ 的光垂直入射到上板。在空气劈尖厚度为 $t$ 的地方,如果 $t$ 等于下列哪个值,你将看到亮条纹?
        \begin{itemize}
            \item[] Α. $\lambda/2$
            \item[] Β. $3\lambda/4$
            \item[] C. $\lambda$
            \item[] D. A 或 C
            \item[] E. A、B 或 C 中的任何一个
        \end{itemize}
\end{enumerate}

\section*{思考题答案}
\begin{enumerate}[label=\textbf{Q\arabic*.答案:}, leftmargin=*]
    \item \textbf{D}
        \begin{itemize}
            \item \textbf{解析}:亮条纹的程差条件为 $\Delta r = m\lambda$。已知 $\Delta r = 1320 \text{ nm}$,$\lambda = 660 \text{ nm}$。所以 $m = \frac{\Delta r}{\lambda} = \frac{1320}{660} = 2$。如果 B 是中央最大值 ($m=0$),那么 C 是 $m=1$ 的亮条纹,D 是 $m=2$ 的亮条纹。
        \end{itemize}
    \item \textbf{A}
        \begin{itemize}
            \item \textbf{解析}:对于 $m=+3$ 的亮条纹,其定义就是两束光到达该点的程差为 $3\lambda$。
        \end{itemize}
    \item \textbf{B}
        \begin{itemize}
            \item \textbf{解析}:亮条纹的位置由 $y_m = R \frac{m\lambda}{d}$ 给出。如果波长 $\lambda$ 增加,而其他条件不变,则 $y_m$ 会增加,这意味着亮条纹之间的间距变大,亮区离得更远。
        \end{itemize}
    \item \textbf{B}
        \begin{itemize}
            \item \textbf{解析}:空气劈尖(玻璃-空气-玻璃)。光线在空气膜上表面反射时(玻璃$\to$空气,$n_{玻璃} > n_{空气}$),无相位跃变。光线在空气膜下表面反射时(空气$\to$玻璃,$n_{空气} < n_{玻璃}$),有$\pi$相位跃变。因此,两束反射光之间有一次净的$\pi$相位跃变。
            \item 对于亮条纹(相长干涉),总光程差应为 $(m + \frac{1}{2})\lambda_{空气}$。几何程差为 $2t$。所以 $2t = (m + \frac{1}{2})\lambda_{空气}$。
            \item 当 $m=0$ 时,$2t = \frac{1}{2}\lambda \implies t = \frac{\lambda}{4}$。
            \item 当 $m=1$ 时,$2t = \frac{3}{2}\lambda \implies t = \frac{3\lambda}{4}$。
            \item 当 $m=2$ 时,$2t = \frac{5}{2}\lambda \implies t = \frac{5\lambda}{4}$。
            \item 选项中 B ($3\lambda/4$) 符合条件。
        \end{itemize}
\end{enumerate}

\end{document}